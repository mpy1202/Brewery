\documentclass[12pt,a4paper,twocolumn]{article}

\usepackage{graphicx}

\begin{document}

	\title{Extraction and pH Stability of Anthrocyanins From Clitoria Ternatea for use as a Beer Colourant - A Review}
	\author{M.P. Young}
	\date{\today}
	\maketitle
	\abstract{}
	\section{Introduction}
	Use of colouring in food has been used to enhance and alter the appearance of the product extensively.
	This colouring has the ability to enhance or add another dimension to the experience of the foodstuff. %TODO: Cite food coloring paper + expand on what i mean
	Ever-wary consumers are looking for alterniatives to synethetic colouring for food products due to the ideas that this can have adverse side-effects on health %TODO: Cite adverse health artifical colouring
	Plant derived alternatives are available and may offer other benifits such as environmental susutainablilty and positive health benefits such as antioxidants.
	The blue flower of the Butterfly Pea (\textit{Clitoria Ternatea}) has a high concentration of pigments known as anthrocyanins, %check latin spelling
		from the greek \textit{anthros} meaning plant based and \textit{cyan} being blue.  %check root word spelling and origin language
	\begin{figure}[h]
		\centering
		\includegraphics[width=0.75\columnwidth]{img/Clitoria_ternatea}
		\caption{Blue flower of the \textit{Clitoria Ternatea} plant}
		\label{flower}
	\end{figure}
	The vivid blue colour of the plant seen in \ref{flower}. 
	However an extraction of the pigment is sensitive to pH and as such the colour can be manipulated in order to create a colour changing effect.
	In future it is planned to use the flower pigment as a novel colour changing pigment in beer.
	This review aims to detail the best extraction methods of the pigments in order to create a concentrated tincture, alongside what pH range the colour change
	is expected between and the colours that will be expected.

	\section{Extraction}
	
\end{document}